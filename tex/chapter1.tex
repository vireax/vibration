%\addcontentsline{toc}{chapter}{Introduction}
%\phantomsection
\chapter{Introduction}
\thispagestyle{empty}
%\emph{intro of intro}

\section{Background}

Genetic Algorithms were invented for different purposes from evolutionary strategies \cite{melanie99}. The mechanisms of natural adaptation reveal natural genetic operators and natural selection. Based on genetic operators such as crossover, mutation and inversion, individuals within a generation interact to reproduce hopefully offspring who will fit better to the environment. This theoretical foundation lead to successive adaption into computational evolution, and therefore various type of nature-inspired algorithms. \emph{No free lunch theorems for optimization} \cite{wolpert1997} suggested that no universally better algorithms exist because the best and efficient algorithms are subject to specific tasks only \cite{yang2010}. This 
%surprise 
theorem encourages researchers to improve their model as well as to cover possible range of application.
%In engineering, ones focus mainly on the efficient use of resource to invent and improve every design, from concept to the end of product life cycle. At the same time, regular maintenance is required to ensure that the system is ... at specified range of performance, as well as to prevent catastrophic breakdown. There are certain range of application where casual maintenance or human access are not possible. Underwater ..., aerospace, extreme temperature and pollution are such life threatening examples. Unconventional techniques have been conceived such as .... However, there are still many rooms for improvements.

%[Vibration based damage detection: How it works? Who did what?  What is missing in the crowd? ]

Vibration based damage detection is a non destructive testing technique usually employed in mechanical and civil applications. Modal properties of the structure such mass, stiffness, and damping are collected for analysis. Damage identification can be classified into different levels such as: presence of damage, location, severity, and prediction of remaining service life; but not limited to classification based on measurement method, instrumentation \cite{doebling1998}. The computation usually involves highly modal function, thus requires expensive exploration in search space. The answer in optimizing this kind of problem is therefore evolutionary approach because despite limited ability to search optimum design variables, nature-inspired algorithms has shown much result of 
%our present days 
up to present \cite{yang2010}.

% How about co-operative co-evolution? Proposed in 1994, 
Not limited to Genetic Algorithms and successive evolutionary techniques, co-operative co-evolution \cite{potter1994} showed significant improvement by evolving within and between species in each generation. 
%This co-operative approach suggested lower computational cost 
However, future implementation of this techniques require the verification of interdependencies between objective function variables. 
 . . . \emph{more: strength and weakness of CCGA} . . .

Diverse range of algorithms and applications leads therefore the main purpose of this research. Although the efficiency of every optimization algorithms will not be studied for comparison, the author will verify in depth the efficiency of co-operative co-evolutionary genetic algorithm on simulated damages of truss structures against previous research using the closest test possible.

\section{Objective}

This research aims to predict level of damage in truss structure using co-operative co-evolutionary genetic algorithms based on the following specific objectives:

\begin{itemize}[leftmargin=\dimexpr 14pt+0.63in]
%[leftmargin=\dimexpr 26pt+0.63in] 14 pt fit better
\item To compare the accuracy of damage detection with previous research based on the principle of residual force and CCGA

\item To predict damage level while including experimental noise during measurement of modal properties
\end{itemize}


\section{Significance}

This research will fill the gap between the integration of theoretical algorithms with application-specific knowledge bases and practical experimental constraints.

\begin{itemize}[leftmargin=\dimexpr 14pt+0.63in]
\item The accuracy of CCGA over traditional GA allows improved technique in solving optimization problem.

\item Verification with experimental noise allows extension in practical use of the system rather than computer simulation and laboratory.
\end{itemize}

\section{Scope}
\begin{itemize}[leftmargin=\dimexpr 14pt+0.63in]
\item The concept of cooperative coevolution will be implemented on conventional genetic algorithms only, other family of evolutionary algorithms will not be tested
\item Planar and space trusses will be modeled based on finite element methods to ensure that any error of the solution are only caused by the algorithm itself, not the test subject
\item Damage index of truss element will be numerically simulated, so as the measurement of modal properties of the damaged structure
\end{itemize}

\section{Literature review}
%REVIEW: Methods, Subjects, Comparison, Parameters, Objectives \cite{doebling1998} and \cite{perera2009} and \cite{rao2004} and \cite{potter1994}

%\emph{integration of theoretical algorithms with application-specific knowledge bases and practical experimental constraints}

Vibration based damage detection techniques use changes in the modal properties (frequencies, mode shapes, and modal damping) caused by changes in the physical properties \cite{doebling1998}. However, slow adoption in industry is caused firstly by inaccuracy in data compression and loss in data transfer of multiple modes. The second obstacle of industrial adoption is caused by the level of sensitivity to global or local response which depends on the level of excitation frequencies modes.

In practice, damage identification depends on prior analytical model or prior test data. Finite Elements Methods (FEM) are expected to minimize this dependence but they may not be appropriate in every case. Actual measurements require correct placement of sensors for sufficient level of sensitivity, repeatability of the tests, and it is also possible to use vibration induced by ambient environmental \cite{koo2008} or operating loads for the assessment of structural integrity. Some data set are available for researchers to verify their methods with real cases but not many works have done sufficient comparison.

Genetic algorithms \cite{holland1975} have been long implemented from biological evolution to various fields, and therefore, in engineering optimization along with other meta-heuristic applications \cite{melanie99}. Damage identification has also adopted this technique to solve mechanical and civil problems in beams \cite{rao2004} and structures \cite{hu2001, nobahari2011}, in either simulation or real models. Scarcity of scopes also enables hybrid methods to improve efficiency \cite{meruane2011} as well as performance assessment between different techniques \cite{perera2009}. 

\citet{potter1994} extend the abstract model of Darwinian evolution and biological genetics by modeling the coevolution of cooperating species. Better performance over conventional genetic algorithms suggests therefore potential representation and resolution of higher complexity problems. 
%while level of interdependencies between objective functions variables require modification of credit assignment algorithms during selection process (detail in section \ref{sec:ccga}).
Recent damage detection in beam \cite{panigrahi} is the core motivation of the second objective. Experimental noise possibly defects the prediction of damage but this problem can never be avoided in practice; examples are irregularity in the body, thermal stress, instrumentation error, etc. If the prediction gives a higher level of accuracy, it should also return better prediction even if are introduced experimental noises. The accuracy of this case clearly cannot be deducted from the performance of the damage identification algorithm alone, CCGA of course, because measurement error occurred earlier than the identification process. However, The algorithm should cover an acceptable range of noise level in order to predict the damage reliably.
